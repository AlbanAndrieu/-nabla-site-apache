%\documentclass[11pt,a4paper,sans]{moderncv}
\documentclass[11pt,a4paper]{moderncv}
%\documentclass[11pt,a4paper,colorlinks,linkcolor=true]{moderncv} 

%\moderncvstyle[blue]{banking} % [blue]{banking} % casual banking fancy
\moderncvstyle{banking}
\moderncvhead{1}

\moderncvcolor{blue}

\renewcommand*{\namefont}{\fontsize{20}{20}\mdseries\upshape}
\renewcommand*{\titlefont}{\fontsize{22}{20}\mdseries\upshape}

\usepackage[utf8]{inputenc}
\usepackage[scale=0.8]{geometry}
%\usepackage{tabularx}


\firstname{Alban}
\familyname{ANDRIEU}
\title{DevSecOps}
%\title{Senior software engineer}

\address{11, terrasse de l'université}{92\,000 Nanterre}{France}
\phone[mobile]{+33 6~95~43~53~53}
\email{alban.andrieu@free.fr}
\homepage{https://cv.albandrieu.com/}
\social[linkedin]  {nabla}
%\social[github]{AlbanAndrieu}
%\photo[100pt][0pt]{alban-andrieu-19-12-2017.png}   

\begin{document}
	
\makecvtitle

\section{Expérience professionnelle}


\cventry{Depuis 2007}{Éditeur de logiciel financier}{Finastra (Ex Thomson Reuters)}{Paris}{}{}
{
\bigskip
\textbf{Ingénieur intégration Cloud} \hfill 2017-2021
\textit(https://www.finastra.com/solutions/treasury-capital-markets)
\bigskip
	\begin{itemize}%
		\item Transformation digitale vers le cloud  
			\begin{itemize}%
				\item Sécurisation de l'infrastructure de déploiement
				\item Containerization de l'application sous Docker et réaliser le déploiement sous Kubernetes
			\end{itemize}
	\end{itemize}
\bigskip
	\begin{itemize}%
	\item	Projet d’intégration pour unifier l'environnement de CI/CD
		\begin{itemize}%
				\item Définition des processus livraison, ajout de métrique de qualité
				\item Unification de l'infrastructure et du système de CI/CD sous Jenkins
				\item Coordination avec plusieurs équipes pour effectuer builder, tester, packager  et livrer
		\end{itemize}
	\end{itemize}
\bigskip
 \textbf{Build, Release et Integration Ingénieur} : \textit{Introduction de l'intégration continue} \hfill 2012-2017
\bigskip
			\begin{itemize}%
				\item Intégration continue avec Jenkins pipeline et automatisation
                \item Introduction la notion de GitOps via Ansible pour Unix, Windows, Solaris, AIX. OSX
				\item Gestion de multiple applications, en C/C++/ObjC, Java, Javascript. Python
					\begin{itemize}	
						\item C/C++, ObjC
						\item Java, Javascript
						\item Python, perl, Shell
					\end{itemize}				
				\end{itemize}
\bigskip
\textbf{Ingénieur logiciel senior} : \textit{Développement du module stock et comptabilité} \hfill 2007-2012

\bigskip
			\begin{itemize}
				\item Intégration au sein de l’équipe K+TP, progiciel de Back-Office de la suite Kondor+. Architecture web 3 niveaux centrée autour de services métiers C++ et interfacés par du Java J2EE via Corba.
				\item Gestion de position, stocks matière et gestion
				\item Génération automatique des écritures comptables bilan/hors-bilan 
			\end{itemize}
}

\newpage

\cventry{2005--2010}{Société Générale Banque et Investissement via ESN}{SGCIB}{Paris}{}{}
{
	\bigskip
	\textbf{Ingénieur d'étude}
\bigskip
\begin{itemize}
\item Suivi et amélioration de systèmes d’informations (EOLE et GATE) de gestion des opérations d’achat/vente sur titre dans le cadre d’un Back Office
	\begin{itemize}
		\item Évolution du code C / PLSQL / Java
		\item Migration Oracle (de 8i à 10g) et OS de (True 64 à Sun Solaris)
		\item Support niveau 2 et production
\end{itemize}
\bigskip
\item Étude des besoins fonctionnel
	\begin{itemize}
		\item Processus règlement livraison (achat/vente, prêt/emprunt, SWIFT, RELIT, cross border)
		\item Rapprochement avec les dépositaires (évolution des statuts, dénouements, suspens)
		\item La génération automatique des écritures comptables quotidiennes et mensuelles
	\end{itemize}
\end{itemize}
}
\bigskip
\cventry{2004}{Société Générale Banque et Investissement}{SGCIB}{paris}{}{}
{
\bigskip
\textbf{Stage d’ingénieur études et développement}
\bigskip
\begin{itemize}
	\item Suivi et amélioration d’un système d’information de gestion des avoirs titres et espèces dans le cadre d’un back-office
\end{itemize}
}
\section{Formation}
\cventry{1999-2004}{Diplôme d'ingénieur en informatique, option système d'information}{EFREI}{Paris}{}{}

\cventry{2002}{Immersion à l'Université de Birbeck College}{}{Londres, Royaume-Uni}{}{}

\cventry{1999}{Baccalauréat scientifique, spécialité physique-chimie}{Lycée ITEC Boisfleury}{Grenoble}{}{}

\section{Langues}
%\cvitemwithcomment{Français}{Langue maternelle}{}
\cvitemwithcomment{Anglais}{Courant}{Environnement international (Londres, Pologne, Inde). TOEIC 750} 
\cvitemwithcomment{Allemand, Norvégien}{Connaissance}{} 

\section{Compétences informatiques}
\bigskip

\cvitem{Méthodes}{ Méthodes agiles (Scrum, kanban), UML, Design patterns}
\cvitem{Langages }{\textbf{C/C++, ObjC, Java, Javascript, PLSQL, Python}, Perl, Shell}
\cvitem{Outils}{\textbf{Docker, Kubernetes, Jenkins, ELK}, JIRA, GIT} 
\cvitem{Environnement}{Windows, Linux (RHEL, Ubuntu), OSX, Solaris, AIX, VMware}
\cvitem{Cloud}{Rancher, AWS, Azure}
%\cvitem{Certification}{ \href{https://www.scrum.org/user/156131}{\underline{PSM 1}}, 
%	 \href{https://www.youracclaim.com/badges/fe9582c4-2f76-4f03-90fd-896041d397b2}{\underline{Safe Agilist}} }


\section{Centres d'intérêts}
\bigskip
Course à pied, Ski, Sécurité informatique

\end{document}
